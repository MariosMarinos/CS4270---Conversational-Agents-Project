The approach to teaching has been extensively studied throughout the years. Irrespective of a lot of discourse in the field it has been widely established that affective behaviour plays a large role in the effectivity of a learning environment.\\

\noindent In \cite{shechtman2004affective} an extensive study was conducted, comparing a cognitive approach with an affective approach to teaching. Examples of results were a lesser frequency of misbehaviour, peer support, increased likelihood of expressing thoughts or feelings and more. Numerous studies like these have been performed and resulted in unambiguous results similar to the ones described above, examples are \cite{griffith2006educators} \cite{miller2005teaching} \cite{garritz2010personal}.\\

\noindent We think it is important for tutors to be emotionally involved in the process of guiding a student through the acquisition of a new, in this case mathematical, skill. It can be extremely frustrating when this process is cumbersome, especially for the student. The relevance of this project is exactly that, creating a conversational agent that doesn't trade in affection for automation but instead provides both.\\