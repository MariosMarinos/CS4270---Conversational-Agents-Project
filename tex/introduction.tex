Math can be very challenging to master. Taking a personal approach to teaching is important to make sure the student stays engaged and dedicated to mastering the subject. We developed a conversational agent using the Furhat SDK \cite{Furhat} to teach a chosen math skill. Percentages is the math skill that we chose arbitrarily for this project, but that is not important for the purpose of the project and could have been any other math skill as well. The requirements for the agent were: that it has about 3 minutes of dialog flow to talk with the user, that there is some kind of gaze behavior implemented, and that it can detect the affect of the user (frustration for example). In this report we will document the design of our agent that achieves these requirements using affective computation and effective dialog management.\\

\noindent The goal of the agent is to teach a user a math skill while adapting to the emotional behaviour of the user, in this case mainly relating to the frustration and performance levels. The agent will guide the user through the process and tries to give the user a sense of achievement while doing so. Our agent is called Professor Euler. First it introduces itself and confirms with the user that they want to learn about percentages, then it has 2 different explanations, between which it can switch or repeat depending on the user's response. It also has different encouraging voicelines in case the user seems frustrated. After the agent is done with their explanation, it will ask some randomly generated exercises so the user can practice their newly learned skill. The agent keeps track of the users' score, so it can give a summary to the user when he/she is done. We also added some gestures to the agent's behavior to make it more human-like.