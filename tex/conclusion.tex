% \newpage
% summarize results here
In summary, from the findings presented in the previous sections, a few interesting remarks can be made regarding the virtual agent.

Firstly, the virtual agent could be expanded to become a general math tutor that is capable of teaching more math skills, not only percentages. This would be a necessity if the virtual agent is to be used as an actual product, as the target audience is too small if it can only teach one skill. The explanation-exercise-feedback loop is already implemented and can be reused for other math skills. The dialogue flow would need to be expanded for new math skills and some additional states have to be added to let the user switch between math skills.

Secondly, the outcome of the evaluation shows that the emotion-detection system has to be revamped. One future direction could be to apply machine-learning on the users' utterances to give better encouragements based on the users' emotion. This would increase the perceived emotional interdependence. 

Another possibility is to capture the users' emotional and attitudinal states better than it's doing currently and let the virtual agent mirror them. An example of this could be to incorporate data like voice pitch and voice intensity. This would increase the perceived affective understanding between the virtual agent and the user. To increase the co-presence and attentional allocation of the virtual agent, it would be good if the virtual agent is capable of listening to users' utterances while speaking, or halt when the user is speaking. This would also increase the perceived message understanding of the virtual agent as the user doesn't have to repeat their utterances that much.
Overall, there is a lot of potential in this field of research in terms of real-world applications. The technology and research regarding virtual agents in e-learning will hopefully increase in the coming years such that these agents become a viable alternative to private tutors.