In retrospect, there are several aspects in our agent that we either thought could have had a better realisation or instead turned out very satisfactory. In this section, we will review the process of the development of our math tutoring agent, specifically what went according to plan and what ended up not living up to our expectations.\\

\noindent Let's start on the negative side of things. At the start of our project, we were very interested in incorporating emotion recognition for our agent to adapt its’ behaviour according to the users’ supposed emotion. This was very relevant for our context, as a main requirement for the tutor was its dynamic behaviour concerning the frustration of the user. In retrospect, it may have been too ambitious to find such a model that predicts these affects with an accuracy that is satisfying to that end. Instead, we found our model to often fall back to a neutral classification or not recognizing the face itself at all. The model we used was scored with a 63\% accuracy rate using annotated visual input.\\

\noindent A second aspect that turned out to make for a cumbersome experience was discovered in the evaluation phase. Since we as developers were supposedly unconsciously aware of this problem the issue did not reveal itself at an earlier stage. When our subject was interacting with the tutor it came to our attention that the number of retries due to unrecognized speech was significantly higher than we had experienced ourselves. This might be due to the subjects accent or otherwise the aforementioned hypothesis involving our unconscious avoidance of the problem.\\

\noindent There was another issue that was brought to the light in this evaluation. During the installation of all the different components of our system, we noticed that there were quite a lot of independent components needed specific software versions in order to integrate with the whole. To tackle this issue, we have written instructions to set up all different components.\\

\noindent Finally, let’s touch on some of the aspects that we are proud of. We think what makes our tutor so useful is that it essentially takes a process that could very simply be individualized and creates a tutoring experience around it. Apart from that, it is not just a math tutor but can be any kind of tutor as the source code is easily adaptable to incorporate new domain knowledge and even allows for migration to a completely different domain altogether. The robustness of the agent allows the initiative of the dialogue to stay with the agent and provides the possibility for a directed and focused approach to learning in this setting.